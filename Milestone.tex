\documentclass{article}
\usepackage{graphicx} % Required for inserting images
\usepackage{amssymb}
\usepackage{amsmath}
\usepackage{bbm}
\usepackage{xcolor}

\title{DQN and DDPS in Loma Milestone}
\author{Ziyuan Lin}
\date{2025.5.26}

\begin{document}
\maketitle
\section{Data Preprocessing}
I have downloaded the Arcade Learning Environment(ALE), which is an environment for game simulation. The frames are captured dynamically through playing.
I reduced the input dimensionality to $84 \times 84$. The frames byte streams flowing into the loma pipeline. I use four frames as a stack to the loma.
The evaluation code is also available now.
\section{Loma Part}
I implemented the matrix operation already. The differentiation code is generated. Since all normal functions are implemented in Loma, I do not need to consider about the
non-linear function implementation. 
\section{Next Step}
The next step will be convolution, which will be more difficult to implement. Then I need to stack the network and start training and inference.


\end{document}